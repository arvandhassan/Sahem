% @Author: Taha Bouhsine


%%%%%%%%%%%%%%%%%%%%%%%%%%%%
% CHAPTER                  %
%%%%%%%%%%%%%%%%%%%%%%%%%%%%
\setcounter{mtc}{10}

\chapter{Realization, GUI And Tests}%
\label{chap:chapter_four}
\minitoc

\section{Hardware environments}
\section{Development environments}
\subsection{Conception}


\paragraph{Gantt chart}
A gantt chart is a horizontal bar chart that visually represents a project plan over time. Modern gantt charts typically show us the status of—as well as who’s responsible for—each task in the project.

In other words, a gantt chart is a super-simple way to keep us out of a project pinch!

What are the key parts of a gantt chart?
A gantt chart is made up of several different elements:
\begin{enumerate}
      \item
            Task list: Runs vertically down the left of the gantt chart to describe project work and may be organized into groups and subgroups
      \item
            Timeline: Runs horizontally across the top of the gantt chart and shows months, weeks, days, and years
      \item
            Dateline: A vertical line that highlights the current date on the gantt chart
      \item
            Bars: Horizontal markers on the right side of the gantt chart that represent tasks and show progress, duration, and start and end dates
      \item
            Milestones: Yellow diamonds that call out major events, dates, decisions, and deliverables
      \item
            Dependencies: Light gray lines that connect tasks that need to happen in a certain order
      \item
            Progress: Shows how far along work is and may be indicated by \% Complete and/or bar shading
      \item
            Resource assigned: Indicates the person or team responsible for completing a task
\end{enumerate}


\subsection{Design}
\paragraph{Adobe Photoshop}
Adobe Photoshop is a software application for image editing and photo retouching for use on Windows or MacOS computers. Photoshop offers users the ability to create, enhance, or otherwise edit images, artwork, and illustrations. Changing backgrounds, simulating a real-life painting, or creating an alternative view of the universe are all possible with Adobe Photoshop. It is the most widely used software tool for photo editing, image manipulation, and retouching for numerous image and video file formats. The tools within Photoshop make it possible to edit both individual images as well as large batches of photos.
We used it to prototype and create our platform logo.
\paragraph{Adobe XD}
Adobe XD is a vector-based user experience design tool for web apps and mobile apps, developed and published by Adobe Inc. It is available for macOS and Windows, although there are versions for iOS and Android to help preview the result of work directly on mobile devices. XD.












\subsection{Development}
\paragraph{Visual Studio Code}
DescriptionVisual Studio Code is a source-code editor developed by Microsoft for Windows, Linux and macOS. It includes support for debugging, embedded Git control and GitHub, syntax highlighting, intelligent code completion, snippets, and code refactoring.


\paragraph{MongoDB Compass Community}
MongoDB Compass is the defacto GUI tool for MongoDB much like MySQL Workbench is MySQL’s associated tool. It allows us to visually explore our data, run ad hoc queries, interact with our data with full CRUD functionality, as well as view and optimize our queries’ performance.



\paragraph{Postman}
Postman is an interactive and automatic tool for verifying the APIs of our project. Postman is a Google Chrome app for interacting with HTTP APIs. It presents us with a friendly GUI for constructing requests and reading responses. It works on the backend, and makes sure that each API is working as intended.

In Postman, we create a request, and Postman looks at the response to make sure it has the element we want in it. As it is an automation tool, it drastically improves testing time and quality of the project. It helps in the early detection of bugs that might sprout at later stages and cause more damage to the system.

Postman is the way to streamline the process of API testing. All APIs that we create and deploy first rigorously go through Postman so that any major or show stopper bugs are identified on time and fewer bugs leak through to later stages.



\paragraph{Git}
% Git is a version control system for tracking changes in computer files and coordinating work on those files among multiple people. It is primarily used for source code management in software development, but it can be used to keep track of changes in any set of files. As a distributed revision control system it is aimed at speed, data integrity, and support for distributed, non-linear workflows.
Git is a distributed revision control and source code management system that
allows several people to work on the same codebase at the same time on different
computers and networks. These can be pushed together, with all changes stored and
recorded. It’s also possible to roll back to an earlier state if necessary.
\paragraph{Github}
At a high level, GitHub is a website and cloud-based service that helps developers store and manage their code, as well as track and control changes to their code. To understand exactly what GitHub is, we need to know two connected principles:
\begin{enumerate}
      \item Version control
      \item Git
\end{enumerate}

\paragraph{Github Desktop}
GitHub Desktop is a fast and easy way to contribute to projects from Windows and OS X, whether we are a seasoned users or new users, GitHub Desktop is designed to simplify all processes and workflow in our GitHub. GitHub Desktop is an open-source Electron-based GitHub app. It is written in TypeScript and uses React.


\paragraph{Boost Note}
% Boostnote is an Open source note-taking app for programmers.
% Boostnote is niche tool because designed for programmers, but we are passionate for it.
% It focuses on writing Markdown note and code snippet quickly, can organized in a better way.
% You can sync data to multi-devices(Mac, Windows, Linux, Android and iOS) via Dropbox.
% Boostnote is not an app suitable for everyone, it ‘s a handy note-taking app for programmers.
% Boostnote has two main features.
% Markdown note
% Since Boostnote is a Markdown editor, mainly write with markdown.
% Content is automatically saved while editing notes.
% Since preview could be viewed with one touch, you can immediately check the Markdown preview you are writing.
% Snippet note
% In the code snippet, you can highlight code syntax in over 100 languages ​​such as Javascript, Python, HTML, etc., and save multiple code snippets in one note.
% The indent and tab size can be set from the editor window.


\paragraph{Latex}
\latex{} is a tool used to create professional-looking documents. It is based on the WYSIWYM (what we see is what we mean) idea, meaning we only have focus on the contents of our document and the computer will take care of the formatting. Instead of spacing out text on a page to control formatting, as with Microsoft Word or LibreOffice Writer, users can enter plain text and let LATEX take care of the rest.
\paragraph{MiKTex}
% MiKTeX provides the tools necessary to prepare documents using the TeX/LaTeX markup language, as well as a simple tex editor: TeXworks.

\paragraph{Tex Live}
% TeX Live is intended to be a straightforward way to get up and running with the TeX document production system. It provides a comprehensive TeX system with binaries for most flavors of Unix, including GNU/Linux, macOS, and also Windows. It includes all the major TeX-related programs, macro packages, and fonts that are free software, including support for many languages around the world. Many operating systems provide it via their own distributions.



\paragraph{Markdown}
Markdown is a lightweight markup language with plain-text-formatting syntax. Its design allows it to be converted to many output formats, but the original tool by the same name only supports HTML. Markdown is often used to format readme files, for writing messages in online discussion forums, and to create rich text using a plain text editor.
\paragraph{Marp}
Marp is the ecosystem to write our presentation with plain Markdown
\section{DBMS Choice}
\subsection{MongoDB}
\subsection{Justification of the choice}

\section{Platform security}
\subsection{Physical level}
\subsection{Logical level}

\section{Interfaces}
\section{tests}
\section{Deployment}


